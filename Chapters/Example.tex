\chapter{ПРИКЛАД НАЗВИ РОЗДІЛУ, ЯКИЙ ПИШЕТЬСЯ ВЕЛИКИМИ ЛІТЕРАМИ. СЛОВО "РОЗДІЛ" ТА НОМЕР ПИСАТИ НЕ ПОТРІБНО}


\section{Назва секції, буде відображена у змісті звіту}

Звичайний параграф тексту, що містить тільки текст. Звичайний параграф тексту, що містить тільки текст. Звичайний параграф тексту, що містить тільки текст. Звичайний параграф тексту, що містить тільки текст. Звичайний параграф тексту, що містить тільки текст.

Звичайний параграф тексту, що містить текст та посилання. Звичайний параграф тексту, що містить текст та посилання. Звичайний параграф тексту, що містить текст та посилання. Текст після якого йде посилання \cite{Chen2019}. 


\section{Інша назва секції}
\subsection{Інша назва субсекції}

\noindent Звичайний параграф тексту, без відступу червоної строки. Звичайний параграф тексту, без відступу червоної строки. Звичайний параграф тексту, без відступу червоної строки. Звичайний параграф тексту, без відступу червоної строки. Звичайний параграф тексту, без відступу червоної строки. % \noindent - знімає відступ в параграфі тексту

Строка із символом переносу каретки \\ % символ "\\"  це кінець строки або перенос каретки

\noindent \textbf{Жирний текст} \\
\noindent \textit{Курсивний текст} \\

\section{Вставка іллюстрацій}

Звичайний параграф тексту, в якому міститься згадування про малюнок. Звичайний параграф тексту, в якому міститься згадування про малюнок. Звичайний параграф тексту, в якому міститься згадування про малюнок. Звичайний параграф тексту, в якому міститься згадування про малюнок. (Рис. \ref{fig:NameOfPicture}) % Назва малюнку (нумерація відбувається автоматично)
Звичайний параграф тексту, в якому міститься згадування про малюнок. Звичайний параграф тексту, в якому міститься згадування про малюнок. 


% блок вставки малюнка
\begin{figure}[h]
\caption{Заголовок малюнку із посиланням \cite{Chen2019}}
\centering
\includegraphics[width=0.9\textwidth]{Illustrations/Logo.png}
\label{fig:NameOfPicture} % Назва малюнку
\end{figure}
% кінець блоку вставки малюнка

Звичайний параграф тексту, в якому міститься згадування про малюнок. (Таб. \ref{table:NameOfTable}) % Назва малюнку (нумерація відбувається автоматично)
Звичайний параграф тексту, в якому міститься згадування про малюнок. Звичайний параграф тексту, в якому міститься згадування про малюнок. 
% Please add the following required packages to your document preamble:
% \usepackage{graphicx}



\section{Нумеровані та ненумаровані списки}

Це нумерований список

\begin{enumerate}
    \item Позиція 
    \item Інша позиція
    \item Ще одна позиція
\end{enumerate}

\vspace{2em} % це відступ

Це ненумерований список

\begin{itemize}
    \item Позиція 
    \item Інша позиція
    \item Ще одна позиція
\end{itemize}
    