\chapter{Малюнки та таблиці}

\section{Перелік необхідних таблиць}

\begin{enumerate}
    
    \item Первинна локалізація ураження у пацієнтів з ГБ (по сегментах, PRETEXT, частота для загальної групи пацієнтів) 
    \item Патоморфологічні варіанти ГБ 
    \item Режими та Ефективність пхт (частота конверсії в резектабельні форми, частота редукції позаорганних проявів) 
    \item Розподіл антропометричних показників (вік, вага, зріст) та коморбідності в загальній виборці та по группах.  ???
    \item Характеристика ураження печінки в загальній виборці та по группах 
    \item Характеристика донорів в трансплантаційній групі (антропометричні показиники, коморбідність) 
    \item Критерії підбору донорів -----
    \item Характеристика оперативних втручань та судинних резекцій в резекційній групі 
    \item Характеристика оперативних втручань в трансплантаційній групі 
    \item Порівняльна характеристика ранніх післяопераційних ускладненнь та летальності 
    \item Причини післяопераційної летальності
    \item Порівняльна характеристика післяопераційної печінкової недостатності. ??????
    \item Порівняльна характеристика віддалених результатів лікування ГБ 
    
\end{enumerate}

\section{Перелік необхідних малюнків}

\begin{enumerate}
    \item PRETEXT (з інтернету, замінити підписи) 
    \item CHICS (з інтернету, замінити підписи) 
    \item Алгоритм діагностики гепатобластоми
    \item Дизайн дослідження (2 групи - трансплантація та резекція на основі результатів ПХТ - схема в поверпоінті
    \item Критерії резектабельності гепатобластоми (та покази до трансплантації)
    \item Критерії операбельності гепатобластоми 
    \item Способи реконструкції портального притоку при трансплантації печінки з приводу ГБ
    \item Способи реконструкції венозного відтоку при трансплантації печінки з приводу ГБ
    \item Алгоритм вибору хірургічного способу лікування ГБ
    
    
\end{enumerate}
