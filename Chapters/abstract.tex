% Abstract
\thispagestyle{empty}
\begin{center}
    \MakeUppercase{РЕФЕРАТ}
\end{center}



\noindent Звіт про НДР: \pageref{LastPage} сторінок, \tottab\ табл.,  \totfig\ рис., \total{citenum} джерел \\

\noindent Об'єкт дослідження – гепатобластома.\\

\noindent Предмет дослідження:  хірургічне лікування пацієнтів в гепатобластомою.\\

\noindent Мета проекту:  покращити результати хірургічного лікування хворих на гепатобластому шляхом розробки нових способів резекційного та трансплантаційного лікування гепатобластом в залежності від стадії пухлини. \\

В процесі роботи розроблено алгоритм обстеження та вибору тактики лікування хворих з гепатобластомою, який дозволяє уникнути невиправданих лапаротомій та обрати оптимальний метод хірургічного лікування для пацієнта. Розроблено та впроваджено в клінічну практику алгоритм підготовки хворих з гепатобластомою до радикального оперативного втручання, який дозволяє зменшити кількість післяопераційних ускладнень та післяопераційну летальність. \\

\noindent ПЕЧІНКА, ПЕЧІНКОВА НЕДОСТАТНІСТЬ, ГЕПАТОБЛАСТОМА, РЕЗЕКЦІЯ ПЕЧІНКИ, ТРАНСПЛАНТАЦІЯ ПЕЧІНКИ, PRETEXT.
