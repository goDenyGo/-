\chapter{ОБГРУНТУВАННЯ ЛІКУВАЛЬНО-ДІАГНОСТИЧНОЇ ТАКТИКИ ПАЦІЄНТІВ З ГЕПАТОБЛАСТОМОЮ}
\section{Диференційна діагностики гепатобластом}
У дітей пухлини печінки зустрічаютья як доброякісні так і злоякісні. Для диференційної діагностики беруть до уваги клінічну картину, лабораторні показники та дані візуалізаційних методів. Клінічні дані включають сімейний анамнез  та результати фізикального обстеження (наприклад, множинні шкірні гемангіоми, пірексія) \cite{pmid20938901}. Із лабораторних досліджень у першу чергу звертаємо увагу на  кількість тромбоцитів, маркери вірусних гепатитів, рівень альфа фетопротеїну. Візуалізаційні методи дослідження допомагають виявити такі ураження як судинні пухлини та мезенхімальна гамартома, які не потребують лікування. Візуалізація сама по собі не є корисною для точного діагностування злоякісного типу пухлини \cite{pmid20922397}.
\subsection{Диференційна діагностики з доброякісними пухлинами}
Найважливішими доброякісними новоутвореннями печінки є судинні пухлини у дітей, кожна з яких також може з’являтися в іншому місці тіла. Література з цього питання заплутана, і багато авторів не роблять усіх клінічно важливих відмінностей. Дитяча гемангіома є відносно поширеною пухлиною, і майже завжди є поліфокальною або дифузною. Біопсія майже ніколи не потрібна, і єдиним важливим диференціальним діагнозом є метастатична нейробластома \cite{pmid21370433}. МРТ та КТ демонструють характерну картину вузликового, прогресуючого та майже повного доцентрового посилення контрасту. Дитяча гемангіома регресує після періоду збільшення на першому році життя. На жаль, у невеликої кількості дітей, яким спочатку поставили діагноз гемангіома, згодом розвивається ангіосаркома, тому для підтвердження доцільно тривале спостереження.

Другою за важливістю судинною пухлиною є швидкоростуча вроджена гемангіома (rapidly involuting congenital hemangioma, RICH). RICH майже завжди представлений у вигляді одиночної маси з периферичним посиленням та відносно низьким центром ослаблення на КТ. Інші доброякісні судинні новоутворення у дітей, рідко виникають в печінці \cite{pmid21509775}.

При мезенхімальній гамартомі на знімках візуалізуються суміш кістозних та тканинних елементів \cite{pmid21621153}. Фокальна нодулярна гіперплазія (FNH) у дітей має непостійний вигляд на КТ, і її іноді важко діагностувати без біопсії, тому потріно надавати перевагу МРТ з контрастом на основі оксида заліза \cite{pmid21830412}.

\subsection{Диференційна діагностики з ішими злоякісними пухлинами}
Диференціація між первинною та вторинною пухлинами печінки, як правило, не є проблематичним у дітей. Метастатичні захворювання печінки від невиявленої первинної пухлини надзвичайно рідкісні, за винятком нейробластоми.

Найбільш поширені первинні пухлини печінки, такі як гепатобластома та гепатоцелюлярна карцинома, не мають характерних візуальних особливостей. При кожному з цих захворювань досить поширені великі розміри та/або мультифокальність, кальцинація та інвазія судин. Деякі незвичні явища на КТ або МРТ можуть натякати на діагностику рідкісних злоякісних пухлин. Недиференційована (ембріональна) саркома часто представляється як велика, тверда маса з гіперінтенсивністю Т2 при МРТ \cite{pmid22201955}. Епітеліоїдна гемангіоендотеліома зазвичай мультифокальна. Вогнища можуть посилюватися та нагадувати «мішень» а також втягувати капсулу \cite{pmid22648963}.

Фіброламелярна карцинома (FLC) може мати центральний фіброзний рубець, і це можна виявити за допомогою КТ або МРТ. Біліарна рабдоміосаркома, як правило, має внутрішньопротоковий характер росту \cite{pmid22648979}.

\section{Хіміотерапія у пацієнтів з гепатобластомою}
До початку терапії і перед плануванням будь-якого оперативного втручання пацієнт повинен бути проконсультований дитячим онкологом, дитячим хірургом (для прийняття рішення про об’єм оперативного лікування) та анестезіологом.
Всім пацієнтам дитячого віку з діагнозом гепатобластома вибір лікування проводився з урахуванням ризику-користі. Група ризику визначалася в залежності від прогностичних факторів: віку пацієнта; рівня АФП, розповсюдження пухлинного процесу в печінці по системі PRETEXT; додаткових критеріїв PRETEXT (ураження пухлиною першого сегменту печінки, втягнення магістральних судин (портальної вени та її гілок, нижньої порожнистої вени, печінкових вен), позапечінкового розповсюдження, кількості вогнищ ураження в печінці, розрив пухлини, наявності регіонарних і віддалених метастазів); морфологічного варіанта будови пухлини. На основі аналізу факторів ризику проводиться стратифікація на три групи ризику: групу стандартного ризику; групу високого ризику; групу дуже високого ризику\cite{pmid24759227}.

Кінцевою метою лікування є повне хірургічне видалення пухлини, що, у свою чергу, є обов'язковою умовою одужання. Проведення передопераційної хіміотерапії може сприяти зменшенню розмірів пухлини та метастазів, а також контролювати можливі мікрометастази. Крім того, передопераційна терапія дає можливість підготуватися до відтермінованого хірургічного втручання\cite{pmid31584686}. Для визначення групи ризику та прийняття рішення проводиться консиліуму у складі: дитячого онколога, лікаря-рентгенолога та дитячого хірурга, та, у випадку розповсюдження PRETEXT III-IV, обов'язкова консультація пацієнта в трансплантаційному центі.

\subsection{Хімітерапія у дітей з гепатобластомою групи стандартного ризику}
До групи стандартного ризику відносяться діти з локалізованою гепатобластомою, з розповсюдженням по системі PRETEXT I, II або III при відсутності додаткових несприятливих  критеріїв, таких як: Низький рівень  АФП (<100 нг/мл); залучення магістральних судин, що відповідає V3 або P2; поширення за межі капсули печінки; розрив пухлини; віддалені метастази\cite{pmid31155201}.    

Пацієнтам із ГБ групи стандартного ризику рекомендовано проведення лікування за протоколом SIOPEL-3 SR (стандартний ризик): на першому етапі проведення передопераційної терапії препаратом цисплатин 80 мг/м2 в дні 1, 15, 29, 44.

Під час передопераційної хіміотерапії пухлинна відповідь визначатиметься за допомогою оцінки рівня АФП щотижня та візуалізаційних досліджень (УЗД після другого та четвертого введення цисплатину). Якщо після двох введень цисплатину не відбувається стабілізації рівня АФП та/або відзначається прогресування пухлинного процесу (збільшення розміру вогнища або вогнищ, збільшення рівня АФП), пацієнтам показано проведення більш інтенсивної терапії в межах рекомендацій для пацієнтів групи високого ризику\cite{pmid9494762}.

Пацієнтам з ГБ групи стандартного ризику рекомендовано в межах лікування за протоколом SIOPEL-3 SR (стандартний ризик) після передопераційної хіміотерапії виконання відтермінованої радикальної операції, ціллю якої є повна резекція первинної пухлини. 

Як тільки стан дитини нормалізується після операції, рекомендовано проведення ад'ювантної хіміотерапії цисплатином 80 мг/м2 в дні 1 і 15.

Програма терапії пацієнтів групи стандартного ризику передбачає лише 6 введень цисплатину. Якщо пацієнт отримав 4 введення цисплатину перед операцією, він повинен пройти 2 післяопераційні курси цисплатину, а якщо перед операцією було проведено 6 курсів цисплатину, то після операції хіміотерапія не призначається\cite{pmid32843604}.

\subsection{Хімітерапія у дітей з гепатобластомою групи високого ризику}
До групи високого ризику відносятьють пацієнти з розповсюдженим ураженням печінки PRETEXT IV або PRETEXT III та втягненням магістральних судин.

Пацієнтам з ГБ групи високого ризику рекомендовано проведення лікування за протоколом SIOPEL-3 HR (високий ризик) з використанням 10 курсів хіміотерапії в альтернуючому режимі та в поєднанні з відтермінованою радикальною операцією\cite{pmid28921939}.

На різних фазах терапії проводиться оцінка пухлинної відповіді, змін в розмірах пухлини з оцінкою резектабельності і/або статусу ремісії відповідно до рекомендацій, описаним нижче. 
Пацієнтам з ГБ групи високого ризику рекомендовано в рамках лікування по протоколу SIOPEL-3 HR (високий ризик) на першому етапі проведення передопераційної терапії препаратами цисплатин 80 мг/м2, дні 1, 29, 57 и 85; карбоплатин 500 мг/кг2, дні 15, 43 и 71; доксорубіцин 60 мг/м2, дні 15, 43 и 71.

Пацієнтам з ГБ групи високого ризику після ад'ювантної хіміотерапії рекомендовано в рамках лікування за протоколом SIOPEL-3 HR (високий ризик) виконання відтермінованої радикальної операції, метою якої є повна резекція первинної пухлини\cite{pmid11819207}.

Відтермінована операція має бути проведена не пізніше 3 тижнів з 85 дня передопераційної хіміотерапії. Проте, за можливості, відтерміновану операцію можна провести після другого введення карбоплатину/доксорубіцину (після 43 дня). Якщо операція неможлива після 85 дня передопераційної фази поліхіміотерапії, але пухлина продовжує відповідати на хіміотерапію, пацієнту проводиться ще максимум два введення карбоплатину/доксорубіцину, які чергуються з одним введенням цисплатину. Можливість радикальної операції оцінюється наприкінці даних додаткових курсів хіміотерапії.
Якщо на 43 день відзначається стабілізація (проведення радикальної резекції залишається неможливим або сумнівним), необхідно розглянути питання щодо трансплантації печінки.

Після операції пацієнтам з ГБ групи високого ризику, як тільки стан нормалізується, показано проведення післяопераційної (ад'ювантної) хіміотерапії карбоплатином 500 мг/м2 на 1 і 29 день (в/в протягом 1 години); доксорубіцин 60 мг/м2 на 1 і 29 день (в/в, 48-годинна безперервна інфузія, тобто по 30 мг/м2/добу протягом двох днів); цисплатин 80 мг/м2 на 15 день (незалежно від гематологічних показників, внутрішньовенна, 24-годинна безперервна інфузія)\cite{pmid28921939}.

Незалежно від часу проведення відтермінованої операції, всі пацієнти отримують однакову кількість курсів і, отже, одну і ту ж загальну кумулятивну дозу хіміопрепаратів.

\subsection{Хімітерапія у дітей з гепатобластомою групи дуже високого ризику}

До групи пацієнтів дуже високого ризику відносяться діти з гепатобластомою з будь-якою стадією поширеності системи PRETEXT, за наявності будь-якого з даних критеріїв: віддалені метастази (як правило, легені); ГБ із низьким рівнем АФП (<100 нг/мл); пацієнтів із спонтанним розривом пухлини.

Пацієнтам з ГБ групи дуже високого ризику рекомендовано перед оперативним лікування проведення лікування за програмою SIOPEL 4, що складається з трьох БЛОКІВ: А1, А2 та А3, які проводяться кожні 4 тижні (тиждень 1, 5 та 9 відповідно). Щоб отримати достатній контроль над пухлиною, всі пацієнти повинні отримати передопераційну заплановану терапію в повному обсязі, навіть якщо пухлина стане резектабельною (можливість повного видалення) до завершення неоад'ювантної програмної хіміотерапії\cite{pmid14966740}.

Передопераційна (неоад'ювантна) хіміотерапія:

БЛОК А1: цисплатин у дозі 80 мг/м2/день в день 1; цисплатин у дозі 70 мг/м2/день в дні 9, 15; доксорубіцин 30 мг/м2/день у дні 8,10. цисплатин у дозі 70 мг/м2/день в дні 29, 37, 43; доксорубіцин 30 мг/м2/день у дні 36, 38. 
БЛОК А2: цисплатин у дозі 70 мг/м2/день в дні 29, 37, 43; доксорубіцин 30 мг/м2/день у дні 36, 38.
БЛОК А3: цисплатин у дозі 70 мг/м2/день в дні 58,64; доксорубіцин 30 мг/м2/день у дні 57, 59.

Під час передопераційних курсів поліхіміотерапії відповідь пухлини визначатиметься після кожного курсу за допомогою оцінки рівня АФП та візуалізаційних досліджень\cite{pmid12461796}. Якщо відбувається прогресування після ініціальної хіміотерапії (як мінімум БЛОК А1), пацієнту слід припинити лікування в рамках даних клінічних рекомендацій та розглянути питання щодо індивідуальної терапії та можливості застосування альтернативних методів терапії\cite{pmid22648963}.

Метою операції є повне видалення пухлини повністю (без мікроскопічних залишків) шляхом часткової або повної гепатектомії. Резекцію пухлини слід проводити відразу після того, як у пацієнта відбудеться відновлення гемопоезу та зникнуть токсичні ускладнення після останнього курсу хіміотерапії\cite{pmid9494762}. 

Наявність метастазів у легенях на момент встановлення діагнозу не є протипоказанням для часткової резекції печінки\cite{pmid18975296}. Легеневі метастази добре реагують на хіміотерапію на фоні терапії може бути досягнутий повний ефект або метастази можуть стати резектабельними до кінця передопераційної хіміотерапії. Видалення резидуальних легеневих метастазів з подальшою резекцією первинної пухлини вважається допустимим та ефективним способом лікування. Зверніть увагу: для трансплантації печінки необхідна санація всіх екстраспечених пухлинних вогнищ пухлини\cite{pmid8749932}.

Пацієнтам з ГБ групи дуже високого ризику з метастазами у легені після БЛОКІВ А1, А2 та А3 або у разі досягнення повного ефекту (підтверджено за допомогою КТ органів грудної порожнини (КТ ОГК)) 
– рекомендовано виконати радикальне видалення первинної пухлини за допомогою часткової гепатектомії або за допомогою трансплантації печінки\cite{pmid15862752}\cite{pmid24362406}.

Пацієнтам з ГБ групи дуже високого ризику з метастазами у легені після БЛОКІВ А1, А2 та А3 або у разі резектабельності метастазів рекомендовано видалення легеневих метастазів та первинної пухлини. Повне видалення метастазів має бути підтверджено за допомогою відповідного візуалізуючого обстеження (КТ ОГК) перед здійсненням резекції первинної пухлини\cite{pmid16045186}.

Пацієнтам з ГБ групи дуже високого ризику, у яких після завершення хіміотерапії блоками А1-А2-А3 досягнуто повної відповіді та витримано радикальне видалення пухлини, рекомендовано після відновлення стану після операції проведення післяопераційної (ад'ювантної) хіміотерапії за протоколом БЛОК З препаратами карбоплатин 500 мг/м2 на день 2, 23, 44 (в/в протягом 1 години) та доксорубіцин 20 мг/м2/добу на день 1, 2, 22, 23, 43, 44 (в/в, 24-годинна безперервна інфузія, сумарна курсова доза 40 мг/м2)\cite{pmid29761829}.

Післяопераційну хіміотерапію слід розпочати, як тільки пацієнт відновиться після операції. Пацієнтам, яким виконано трансплантацію печінки після БЛОКІВ A1 – A3, також показано післяопераційну хіміотерапію, тільки якщо немає яскраво виражених хірургічних або імунологічних протипоказань\cite{pmid27910913}.

Пацієнти з неповною хірургічною резекцією та/або наявністю нерезектабельних позапечінкових вогнищ захворювання вимагають розгляду індивідуальної терапії, подібна ситуація не є чітко стандартизованою та вимагає обов'язково обговорення мультидисциплінарною командою, що спеціалізується на лікуванні ГБ\cite{pmid28347528}. 

\section{Хірургічне лікування пацієнтів з гепатобластомою}
Резекція пухлини залишається основним методом лікування у більшості випадків гепатобластом у дітей. Однак хірургія печінки все ще залишається проблемою для багатьох менш досвідчених центрів, навіть незважаючи на зниження смертності в наш час від 0\% до 5\% \cite{pmid25649007}. У дослідженні SIOPEL 1, в якому брав участь 91 центр із 30 країн, рівень хірургічної смертності становив 4\% (5/115 пацієнтів) \cite{pmid25783395}. Однак часто операції проводяться у менш досвідчених центрах, де зустрічається лише один випадок резекції печінки на 1-2 роки.
\subsection{Біопсія}
Усім пацієнтам проводили ретельне передопераційне обстеження, включаючи біопсію. Три найбільші міжнародні дослідницькі групи мають різні підходи до біопсії у дітей із підозрою на первинні злоякісні пухлини печінки. В останніх протоколах SIOPEL біопсія стала обов’язковою у випадках підозри на гепатобластому, незалежно від розміру та очевидної резектабельності пухлини \cite{pmid25945430}. Раніше Дитяча онкологічна група (COG) та Німецька дитяча група з онкології та гематології (GPOH) рекомендували лапаротомію з первинною резекцією пухлини \cite{pmid26106955}. Цей підхід останнім часом дещо змінився і первинна резекція печінки виконується тільки  при можливості резекції пухлини з домомогою стандартної гемігепатектомії \cite{pmid26835349}. Згідно протоколів COG у всіх інших дітей проводиться діагностична біопсія. Однак, згідно протоколу GPOH, біопсія непотрібна у пацієнтів у віці від 6 місяців до 3 років з однозначними клінічними даними, типовою візуалізацією та високим рівнем альфа-фетопротеїну (AFP) \cite{pmid26945966}. Хоч біопсію можна безпечно пропустити у так званих типових випадках гепатобластоми, на теперишній час її наполегливо рекомендується у всіх випадках. По-перше, це дозволяє безпечно та обгрунтовано використовувати хіміотерапію на основі діагностики пухлинних тканин. По-друге, гепатоцелюлярна карцинома (HCC) може виникати навіть у дуже маленьких дітей, наприклад, у великому дослідженні із США 5 з 28 HCC діагностовано у пацієнтів молодше 5 років \cite{pmid26945966}. HCC також діагностовано у пацієнтів віком від 3 років \cite{pmid27501172}. У маленьких дітей (віком до 12 місяців) фізіологічно підвищений АФП може мати незрозумілий ефект. Крім того, у цій віковій групі частіше зустрічаються доброякісні пухлини, тобто гемангіоми та гемангіоендотеліоми.

Метою біопсії пухлини є отримання тканини для точного діагнозу, та уникнути ускладнень. Загалом біопсія пухлини - це безпечна та діагностично надійна процедура. Ускладнення, пов’язані з біопсією печінки, найчастіше не значні та відносно рідко виникають, (щонайбільше у 5–10\% пацієнтів) \cite{pmid27730288}. Найбільш небезпечним ускладненням є кровотеча. Існує також ризик розповсюдження пухлинних клітин у здорові сегменти печінки, передню черевну стінку або очеревину, хоча ці ускладнення дуже рідкісні і не було виявлено при дослідженнях SIOPEL \cite{pmid27781375}. У дослідженнях SIOPEL 1 та 2, коли у більшості випадків застосовували відкриту біопсію, ускладнень біопсії, що загрожували життю, не зафіксовано. У дослідженні SIOPEL 1 ускладнення мали місце в 6\% випадків (7/122) і, як правило, були незначними: кровотеча з місця біопсії у чотирьох пацієнтів (один відкритий, три закриті), біль у животі у двох (один відкритий, один закритий) та інфекція рани у дитини, якій зробили відкриту біопсію \cite{pmid27910913}. Усі сім пацієнтів повністю одужали при консервативному лікуванні. Не зафіксовано випадків розриву або локального метастазування пухлини. Раніше у дослідженні відкриту біопсію проводили за допомогою діагностичної лапаротомії. Однак ризики можна мінімізувати, використовуючи черезшкірну техніку або (при лапароскопічній біопсії) використовуючи захисну голку для наведення біопсійної голки. Це особливо важливо при підозрі на неоперабельну гепатоцелюлярну карциному, оскільки вона пов'язана з високим ризиком метастазування біоптаційного шляху. 

При теперишньому рівні можливостей візуалізаційних методів дівагностики печінки (КТ та/або МРТ) та наявності черезшкірної біопсії пухлини, необхідність в діагностичній лапаротомії відпала \cite{pmid28126357}. Як правило, не рекомендується цитологія аспірацийної тонкоголкової біопсії через можливу недостатню кількість матеріалу для діагностики та неможливічть зберігати тканини для біологічних досліджень.

Перед біопсією намагалися виправити тяжку коагулопатію або тромбоцитопенію. Біопсія у дітей проводиться під загальним наркозом. Під контролем УЗД у режимі реального часу. Використовуються автоматичні або напівавтоматичні ріжучі голки (16 або 18 калібру), які забезпечують досить багато тканини пухлини \cite{pmid28138611}. По можливості зовнішню голку слід пропускати через неушкоджену печінку на невелику відстань, щоб мінімізувати можливість розповсюдження пухлини. 

Крім того, біопсію можна проводити під лапароскопічним контролем. Перевагою цього методу є можливість виключення позапечінкових метастазів у черевній порожнині, а також контроль гемостазу в зоні біопсії. Місце біопсії необхідно позначити, для видалення його під час остаточної операції, особливо якщо діагноз гепатоцелюлярна карцинома \cite{pmid28211941}. 

\subsection{Оцінка анатомії та резектабельності печінки}
\subsubsection{Печінкова анатомія}
Основою для можливості виконання резекцій печінки як у дорослих так і у дітей є знання анатомії печінки. Печінка на відміну від більшості органів має подвійне кровопостачання: портальну та артеріальну; однак більшість злоякісних пухлин печінки кровопостачаються переважно артеріальними судинами. На даний час хірургічна спільнота користується загальновизнаною хірургічною анатомією печінки по Куіно (Couinaud), печінка ділиться на вісім сегментів, включаючи сегмент 1, що відповідає хвостатій долі. \cite{pmid28347528}. Ліва печінкова частка складається із сегментів 2–4, а права - сегментів 5–8. Поділ між верхніми та нижніми сегментами проходить по лінії біфуркації ворітної вени по горизонталі, тоді як поділ між печінковими долями проходить по середній печінковій вені або умовній лінії між ложе жовчного міхура та запечінковим відділом нижньої порожнистої вени. Крім того, печінку розділяють на чотири сектори: лівий латеральний (сегменти 2 + 3), лівий медіальний (сегменти 4a і 4b), правий передній (сегменти 5 + 8) і правий задній (сегменти 6 + 8).  поділяють за ходом правої та середньої печінкової вени та умбілікальної фісури або умбілікальної частини лівої ворітної вени. На багатьох анатомічних схемах ліва печінкова вена помилково показана як межа між лівим медіальним та лівим латеральним відділами, тоді як насправді вона проходить ліворуч від цієї межі \cite{pmid2852560}. Частка каудата анатомічно розділена на три області, однак він рідко вражається гепатобластомою \cite{pmid28620649}. 

\subsubsection{Резектабельність}
Оцінка резектабельності пухлини під час встановлення діагнозу, так і після передопераційної хіміотерапії є дуже важливою, так як від цього залежить стратегія подальшого лікування. При цьому важливим є можливість виконувати високоякісні дослідження з контрастом КТ та/або МРТ \cite{pmid28651228}. Також важливим є УЗД з допплерівськими контролем. Це особливо цінним метод для оцінки резектабельності пухлини печінки у дітей, оскільки дозволяє в режимі реального часу досліджувати пухлину та її відношення до печінкових судин і може допомогти візуалізувати судинні структури, які не видно на КТ \cite{pmid28921839}. Крім того за допомогою УЗД також легше розрізнити фактичне ураження судин та наявність пухлинного тромбу лише за допомогою компресії судини. 

У 1990 році група SIOPEL представила передопераційну систему постановки пухлини під назвою PRETEXT (PRE-treatment Tumor EXTension, поширення пухлини до лікування) \cite{pmid28921939}. Вона була спеціально розроблена для прогнозування резектабельності пухлини та моніторингу реакції на передопераційну хіміотерапію. PRETEXT описує кількість задіяних ділянок печінки, а також наявність позапечінкової патології або ураження судин, кодованих додатковими літерами: V, P, E, M). Букви V, P, E та M були додані для венозного, портального, позапечінкового та метастатичного залучення. Незважаючи на те, що позапечінкова та вузлове ураження при гепатобластомі зустрічається рідко, і необхідно  підтвердити в будь-якому випадку біопсією. Ураження судин визначали як наявність внутрішньосудинного тромбу, закупорювання судини або повну оклюзію пухлиною. 

У 2005 році система PRETEXT зазнала важливої модифікації та була деталізована завдяки введенню наступних змін \cite{pmid29286034}:

\begin{enumerate}
    \item Залучення сегмента 1 було включено та кодовано як “C1” (з хвостової частки). Крім того, згідно з визначенням, усіх пацієнтів із С1 слід віднести принаймні до категорії PRETEXT 2.  
    \item При позапечінковому ураженні додають суфікс "а", якщо присутній асцит.
    \item Мультифокальність пухлини була додана і закодована як “F1”.
    \item Розрив пухлини або внутрішньочеревна кровотеча під час діагностики було включено та кодовано як “H1”. Субкапсулярна або пов’язана з біопсією кровотеча не включається і, отже, кодується як “H0”.
    \item Втягнення вузлів було кодовано як "N1", якщо обмежуватись животом. У позачеревній області його кодували як “N2”.
    \item Залучення судинних структур було уточнено і кодовано як «P0», «P1», «P2» (залежно від кількості основних портальних стовбурів) а також “V0”, “V1”, “V2”, “V3” (залежно від кількості вражених печінкових вен та / або IVC). Крім того, суфікс "а" додавали у випадках наявності пухлинного тромбу.
\end{enumerate}
\vspace{2em}

Підтверджено, що система PRETEXT не тільки допомагає оцінити резектабельність пухлини під час діагностики та після передопераційної хіміотерапії, але також визначає прогноз пацієнта \cite{pmid28203111}. Крім того, PRETEXT продемонстрував свою користь у визначенні пацієнтів, які піддаються оперативному хірургічному лікуванню (PRETEXT I та II), а також тих, хто є потенційними кандидатами на трансплантацію печінки \cite{pmid29341393}. В даний час PRETEXT принята усіма основними групами науковців, що займаються хірургією печінки \cite{pmid29375822}.

Іншими факторами, що впливають на резектабельність пухлини, є мультифокальне враження та судинні інвазії: гілки ворітної та печінкових вен, а також НПВ \cite{pmid29761829}.

Завдяки вражаючим регенераторним властивостям печінки можна безпечно видалити до 75–85\% печінкової паренхіми \cite{pmid29888545}. Протягом 6 місяців після основної резекції печінки об’єм печінковго залишку досягає до 80–90\% від початкового об’єму \cite{pmid29906233}. Цей регенеративний процес обмежений у пацієнтів з цирозом печінки та зниженим функціональним резервом печінки. 

Резектабельність пухлини також залежить від хірургічного досвіду оперуючого хірурга. Так деякі великі пухлини, що вражають обидві частки печінки, можуть бути радикально видалені за допомогою розширеної гемігепатектомії (трисегментектомії) за умови, що один бічний відділ печінки залишається не ураженим. І навіть інвазія в запечінковий сегмент НПВ не виключає радикального видалення, оскільки можна виконати резекцію сегмента НПВ і виконати його пластику, або протезування синтетичним (Gore-Tex), або венозним аутотрансплантатом (з використанням внутрішньої яремної або зовнішньої клубової вени) \cite{pmid29906299}.

У разі часткової інвазії стінки НПВ можна використати аутовенозну або синтетичну (Gore-Tex) заплату \cite{pmid30003622}. При залученні в пухлинний процес усіх трьох печінкових вен, одну з них можна реконструювати за допомогою протеза або судинного аутотрансплантата (наприклад, частини резектованої ворітної вени) або, якщо можливо, після резекції накладений анастомоз між печінковою веною та НПВ \cite{pmid30084209}. В деяких випадках, при центральному розташуванні пухлини з залученням усіх печінкових вен можна виконати атипову резекція печінки з видаленням усіх трьох печінкових вен, але це можливе при наявності добре розвинених додаткових печінкових вен, що забезпечують адекватний відтік крові до НПВ \cite{pmid30086577}. У разі недостатнього об’єму прогнозованого печінкового залишку, можна виконати  емболізацію гілок ворітної вени \cite{pmid30270490}.

Беручи до уваги чудові результати первинної ортотопічної трансплантації печінки для гепатобластоми, та значно гірші при спробах трансплантації після резекцій печінки з рецидивами \cite{pmid30528797}, прийшли до висновоку, що слід уникати дуже складних резекцій печінки, які несуть високий ризик рецидиву пухлини, на користь первинної трансплантації печінки. При трансплантаціях печінки при гепатобластомі спостерігається дуже хороша довготривала виживаність в межах 80\% \cite{pmid30577683}. Це особливо важливо, так як її виконують переважно пацієнтам, які не підлягають звичайному хірургічному лікуванню, це переважно пухлини PRETEXT IV. Резекція печінки з видаленням нижньої порожнистої вени та її заміщення протезом Gore-Tex \cite{pmid30762666}.

Трансплантація печінки також рекомендується при пухлинах PRETEXT III, розташованих центрально в безпосередній близькості від основних судинних структур. Це питання дещо дискусійне, враховуючи хороший результат для пацієнтів з мікроскопічним залишком та труднощі, пов’язані з трансплантаційною хірургією, тобто відсутність донорів (частково вирішене використанням живих родинних донорів) та необхідність пожиттєвої імуносупресії \cite{pmid30819543}. 

Раніше лише близько 30–40\% пухлин печінки були резектабельними, але завдяки значному прогресу, на сам перед це використання передопераційної хіміотерапії та трансплантацію печінки, поточний показник резекції гепатобластоми зі стандартним ризиком значно перевищують 90\% (97\% у дослідженні SIOPEL 2 \cite{pmid30946509} \cite{pmid31130716}. Більше того, дослідження SIOPEL показали, що 25\% спочатку неоперабельних пухлин можна перетворити на резектабельні за допомогою звичайних засобів \cite{pmid31155201}. 

\subsection{Резекції печінки}
Кінцевою метою хірургічної операції при первинних пухлинах печінки є досягнення повного очищення пухлини. Згідно рекомендацій SIOPEL перед хірургічним видаленням пухлини необхідно призначення хіміотерапії у всіх пацієнтів, оскільки у більшості розміри пухлини зменшуються, що сприяє резекції печінки \cite{pmid31225421}. Дуже рідко, пухлина під час проведення передопераціної хіміотерапії збільшується в розмірах. У дослідженні SIOPEL 1 це сталося лише у чотирьох пацієнтів (3\% від загальної кількості), і лише один з них став нерезектабельним, пізніше йому виконано \cite{pmid31262438}. Хіміотерапія не збільшує кількість хірургічних ускладнень \cite{pmid31434361}. Клінічні спостереження стверджують, що пухлини після хіміотерапії стають більш твердими, краще відокремленими від навколишньої паренхіми печінки та менш схильними до кровотеч \cite{pmid31584686}. Німецькі дані, зібрані протягом двох послідовних випробувань HB89 та HB94 показали, що первинна гепатектомія була пов'язана із значно більшим числом неповних резекцій у порівнянні з тактикою відстроченої хірургії після хіміотерапії: 30\% (14/48) проти 19\% (15/78) \cite{pmid31645068}. Однак досить багато хірургів із США виконують першочергово резекцію печінки, якщо вона локалізується в одній долі з метою зниження негативного впливу хіміотерапії на паренхіму печінки. За такого підходу близько 40\% пухлин можна видалити одразу, ще 45\% можна успішно оперувати після передопераційної хіміотерапії \cite{pmid31683629}.

Загальновідомо, що лише повна резекція пухлини дає реальну надію на лікування дітей із злоякісними пухлинами печінки, тому всі варіанти лікування, включаючи ортотопічну трансплантацію печінки слід вивчити перед тим, як оголосити пухлину нерезектабельною \cite{pmid31718024}. 

Безпечний край резекції при гепатобластомі є поняттям суперечливим. Досягнути рекомендованого мінімального запасу паренхіми при резекції печінки в 1 см буває досить важко особливо при операціях у дітей. Близьке розміщення пухлини до ворітної вени або печінкових вен не є протипоказом до резекції печінки \cite{pmid31931965}. Допускається відстані в кілька міліметрів від пухлини до краю резекції \cite{pmid32181433}. При мультифокальних пухлинах слід враховувати резекцію повністю всіх інвазованих ділянок печінки, при можливості навіть тих, де локалізувалися вилікувані за допомогою передопераційної хіміотерапії. Якщо є сумніви щодо повноти резекції та виявлено макроскопічно залишки пухлини, хірург неодмінно повинен спробувати виконати повну резекцію. У сумнівних випадках також можна вивчити резекційний край мікроскопічно \cite{pmid32421442}. 

Остаточна оцінка повноти резекції залежить від остаточного патогістологічного заключення та від повернення післяопераційного рівня АФП до нормального стану. Це може зайняти більше 2 місяців, оскільки період напіввиведення циркулюючого АФП становить близько 6 днів, а його рівні перед операцією можуть бути дуже високими. Слід пам’ятати, що мінімальне підвищення рівня АФП незабаром після операції може бути ознакою регенерації печінки \cite{pmid32458263}.
\subsubsection{Технічні аспекти}
Як зазначалося раніше, дитячі пухлини печінки є досить рідкісними, тому для отримання найкращих результатів хірургічного лікування операції слід проводити у спеціалізованих центрах, де є достатній досвід та оснащення, наприклад ультразвуковий дисектор типу CUSA, інтраопераційне УЗД та інші. Крім того, має бути добре оснащені відділення для післяопераційного догляду пацієнтів та досвідчених анестезіологів. 

Інтраопераційний ультразвуковий контроль може суттєво допомогти при резекціях печінки, особливо при мультифокальних гепатобластомах та пухлинах, що розташовані беспосередньо біля основних судинних структур \cite{pmid32603027}, а також при сегментарних резекціях печінки. 

Перший етап операції - це мобілізація печінки з її зв’язокового апарату. Скрупульозне виділення та контроль усіх судинних структур над і під печінкою, включаючи ворітну вену печінки. Для забезпечення найкращих результатів резекцій печінки в переважній більшості випадків виконуються анатомічні резекції печінки з трансекцію паренхіми між сегментами вздовж анатомічних орієнтирів. Це має вирішальне значення для виконання успішної резекції пухлини та мінімізації шансів на ускладнення і забезпечує достатнє кровопостачання та дренаж жовчі у печінковому залишку. Зазвичай спочатку виділяють та пересікають структури воріт печінки, при можливості і печінкові вени, після чого відбувається трансекція паренхіми по лінії ішемії. (рис.). При використанні будь-якої техніки резекції печінки, важливою метою є мінімізація крововтрати та гемотрансфузії, оскільки це доведений фактор рецидиву гепатоцелюлярної карциноми у дорослих \cite{pmid32843604}. 

Для мінімізації крововтрати використовують декілька основних прийомів. Перш за все це підтриманя низького центрального венозного тиску (ЦВТ) під час резекції печінки. При резекція печінки уже стандартом є використання прийому Прингла, тобто перетискання печінково-дванадцятипалої зв’язки. Однак у багатьох випадках можна безпечно виконати резекцію печінки не перекриваючи кровотік, що допомагає зберегти оптимальну післяопераційну функцію печінки \cite{pmid33210501}. Для трансекції паренхіми можна використовувати декілька методів: ультразвуковий аспіратор, водонапірний ніж або техніку краш-клампу.

Для локального гемостазу можуть використовуватися Ligasure, ультразвуковий скальпель, моно- та біполярну коагуляцію, аргон, судинні затискачі, та місцеве застосування тромбостатичних матеріалів або біологічних герметиків. 

У всіх випадках слід проводити лімфаденектомію печінково-дванадцятипалої зв’язки, так як це впливає на прогноз. 

\subsubsection{Анатомічні резекції печінки}
У більшості випадків гепатобластоми зі стандартним ризиком показані типові гемігепатектомії (рис.). Резекція лівої латеральної секції показані досить рідко, при локалізації пухлини виключно в 2 та 3 сегментах печінки. Великі та мультифокальні пухлини, найчастіше обмежені трьома ділянками печінки, і можуть потребувати розширеної гепатектомії (трисекціоектомії) (рис.). Відсоток розширених резекцій може досягати 40\% випадків \cite{pmid33224825}. 

При наявності невеликої пухлини у деяких пацієнтів можна видалити за допомогою сегментектомії або бісегментектомії.
\subsubsection{Атипові резекції печінки}
У пацієнтів з гепатобластоми, як і з іншими злоякісними пухлинами печінки, особливо з ГЦК, слід уникати атипових резекцій печінки, оскільки вони пов'язані з вищим показником неповного видалення пухлини, збільшенням кількості післяопераційних ускладнень, та рецидивом. У двох послідовних дослідженнях із загальної кількості 129 резекцій печінки, 36 із яких були неанатомічними. У 38\% цих випадків були виявлені неповні резекції у порівнянні з лише 18\% у пацієнтів з типовими резекціями печінки. Ця різниця була статистично достовірною \cite{pmid33526266}. Це можна пояснити розповсюдженням пухлинних клітин у печінці по кровоносним судинам при мікроскопічній судинній інвазії при атипових резекціях печінки. Крім того, фактор росту гепатоцитів може стимулювати регенерацію печінки та проліферацію пухлинних клітин. Таким чином, стандартні анатомічні резекції печінки є найкращою тактикою хірургічного лікування \cite{pmid33575308}. 

Атипові резекції печінки виправдані лише у вибраних випадках, головним чином при мультифокальних пухлинах, коли не можна зробити трансплантацію печінки. 

\subsubsection{Високотехнологічні хірургічні методи}

Коли трансплантація печінки недоступна з будь-якої причини, у складних випадках можуть застосовуватися спеціальні методи резекції печінки, особливо при значному враженні судин та/або жовчних протоків. До таких резекцій відносять резекції печінки з тотальною васкулярною ексклюзією, резекції пухлини з повним відключенням печінки з кровотоку та гіпотермічною перфузією, екстракорпоральні резекції з аутотрнсплантацією печінкового залишку \cite{pmid33718305}. За звичай залучення хвостатої долі в пухлинний процес у дітей відбувається досить рідко, однак при резекції печінки з хвостатою долею потребують кращих навичок від хірурга через анатомічні особливості венозного відтоку від сегмента в НПВ через велику кількість коротких печінкових вен. Дуже рідкісні випадки виконання виключно тотальної каудальної лобектомії при локалізації пухлини виключно в межах І сегменту печінки. \cite{pmid34441025}. Повне виключення судин печінки з одночасним затисканням над- та підпечінкових відділів НПВ, а також печінково-дуоденальної зв’язки використовують при ураженні НПВ та необхідності її  подальшої реконструкції. Доведено, що час теплової ішемії можна безпечно продовжувати до 45 хвилин і більше \cite{pmid7754739}. 

В центрах трансплантації доступний метод резекції пухлини з резервною трансплантацією. При цьому печінку можна повністю мобілізувати та видалити печінку, перейшовши таким чином “точки неповернення”, або виконати резекцію пухлини ex-vivo з подальшою аутотрансплантацією печінки, хоча остання методика сьогодні використовується рідко \cite{pmid34464895}.

Останнім часом лапароскопічні методи почали активно застосовувати до пухлин печінки, не тільки у дорослих, а і при доброякісних пухлинах у дітей \cite{pmid7796018} Тривалий час, ризик газової емболії та труднощі при контролі кровотечі, були основними перешкодами для лапароскопічної хірургії печінки \cite{pmid8640025}. На даний час у більшості випадків показами до лапароскопічних резекцій є пухлини, розташовані в легкодоступних передніх сегментах печінки, це II, III, V, VI та частково IV \cite{pmid8749932}. 

Найпростіша і найбільш поширена лапароскопічна анатомічна резекція печінки - це лівобічна латеральна секціоектомія \cite{pmid9149752}. Однак лапароскопічні методи вимагають не лише доступу до спеціалізованого та дорогого обладнання, наприклад, але також великого досвіду в малоінвазивної хірургії.

\subsubsection{Післяопераційні ускладнення та їх лікування}
Післяопераційні ускладнення відносно часті при хірургічних операціях на печінці, і досягають 15–30\% \cite{pmid9149752}. Найбільш частими ускладненнями є: інфекційні ускладнення, інтраопераційна кровотеча, інтраопераційний розрив пухлини, повітряна емболія, післяопераційна жовчотеча або тромбози, транзиторна гіпоглікемія і кишкова непрохідність через спайковий процес \cite{pmid9494762}.
\subsubsubsection{Кровотеча}
У дослідженні SIOPEL 1 було 3 випадки післяопераційних кровотеч серед 115 випадків \cite{pmid9591340}. Часто причиною кровотечі може бути додаткова права або ліва печінкові артерії, які відходять не типово і не були виявлені на доопераційному етапі діагностики. Для зупинки або зменшення інтраопераціїйної кровотечі, якщо не використовували до цього, може допомогти прийом Прингла, а також зниження центрального венозного тиску.
\subsubsubsection{Біліарні ускладнення}
Зовнішня жовчна нориця є одним із частих післяопераційних ускладнеь після резекцій печінки і зустрічається у 2–12\% випадків, і його частота не зменшувалась роками \cite{pmid29888545}. Якщо ця проблема не вирішується за період близько тижня, може знадобитися ендоскопічна ретроградна холангіопанкреатографія (ЕРПХГ) або МРТ-холангіографія (MRCP) з метою виявлення джерела витоку жовчі, яке зазвичай знаходиться на резекційній поверхні печінки. Іноді це може бути наслідком невиявленої аномалії біліарної анатомії або надмірного скелетування головних жовчних проток під час виділення воріт печінки. Щоб запобігти цьому перев'язку жовчних протоків виконють на рівні хілярної пластинки після транссекції паренхіми, уникаючи таким чином будь-якої дисекції позапечінкового жовчного дерева \cite{pmid24759227}. У більшості випадків дефект жовчних шляхів можна керувати консервативно за допомогою тривалого зовнішнього дренажу, а іноді за допомогою додаткової ендоскопічної сфінктеротомії та внутрішнього жовчного стентування \cite{pmid16404555}. Однак, у випадку виділення більше 100 мл на десятий день спостереження має невисокі шансами на успішне консервативне лікування \cite{pmid15285242}. 

При неефективності консервативної терапії виконують гепатикоєюностомію на петлі кишки Roux-en-Y. Витікання жовчі внаслідок пошкодження основних жовчовивідних шляхів може призвести до утворення стриктури в процесі загоєння. В іншому випадку біліарна стриктура може бути результатом надмірної дисекції жовчних проток та їх подальшої деваскуляризації. Це ускладнення є дещо частішим після трансплантації печінки (досягає 20–30\% пацієнтів), ніж після звичайної резекції печінки \cite{pmid16123986}. Незважаючи на те, що у дорослих може застосовуватися ретроградна ендоскопічна або антеградна черезшкірна черезпечінкова дилатація стриктури, досвіду використання цієї методики у дітей мало \cite{pmid12115331}. Таким чином, біліодегестивні анастомози залишаються стандартною хірургічною тактикою.

\subsubsubsection{Інші усклднення}
Іншими ускладненнями, які зустрічаються при хірургії печінки, є: утворення спайок, післяопераційна інвагінація, інфікування рани, формування абсцесу. Зупинка серця може бути наслідком не тільки надмірної кровотечі, а і повітряної емболії, або емболії пухлинного матеріалу. 

\section{Трансплантація печінки при нерезектабельних гепатобластомах у дітей}
Незважаючи на те, що першим пацієнтом з довготривалим результатом виживаності після транпслантації печінки, описаним Starzl у 1968 р., була дитина з атрезією жовчних протоків та випадковою знахідкою - гепатоцелюлярною карциномою \cite{pmid24734315}, пройшло багато років для удосконалення даної методики лікування. Перші довгострокові результати у великої вибірки дорослих з діагнозом гепатоцелюлярною карциномою (ГЦК) були невтішними через високу кількість рецидивів пухлини. Завдяки цьому ранньому досвіду та високому рівню рецидивів пухлин після трансплантації, у 1980-х і на початку 1990-х років трансплантація перейшла у розділ рятувальної терапії; спробувати щось після того, як все інше було використано та не дало результату. Пізніше в 90-х роках, коли ефективна хіміотерапія на основі платини стала загальновизнанем явищем, результати почали покращуватися. Протягом останнього десятиліття трансплантація печінки стала важливим варіантом лікування у багатьох дітей з нерезектабельними пухлинами печінки, зокрема, гепатобластоми \cite{pmid20922397}. Досвід, накопичений у всьому світі є досить об’ємним і найбільш сприятливим для ГБ, які вивчалися в  наступних дослідницьких групах - Дитяча онкологічна групи (Children’s Oncology Group (COG)), Німецька дитяча онкогематологічна група (Pediatric Oncology Hematology Group (GPOH)) та Стратегічна група Міжнародного наукового товариства дитячої онкології (Liver Tumor Strategy Group of the International Society of Pediatric Oncology (SIOPEL)) – вони ж запропонувати керівні принципи, які включають сучасні показання та протипоказання до трансплантації печінки дітям з ГБ \cite{pmid20223320}.

Нетрансплантаційні варіанти хірургічної резекції у цих пацієнтів іноді можливі завдяки досягненням в хірургії печінки та такими методам, як повна судинна оклюзія, гіпотермічна первузія консерваійним розчином in situ та складні венозні резекції та реконструкції порожнистої вени \cite{pmid12461796}. По мірі розвитку даних хірургічних методів, у спеціалізованих центрах печінки, що мають досвід у складних хірургічних операціях на печінці, іноді можлива розширена резекція печінки з реконструкцією судин. У літературі почав з'являтися новий термін, що описує ці героїчні резекції печінки як "екстремальні резекції печінки". Ці „екстремальні резекції печінки” не є обов’язково безпечнішими за трансплантацію, але, оскільки вони розширюють межі технічних можливостей, вони змушують нас чіткіше замислюватися про потенційні ризики та переваги різних варіантів \cite{pmid12115331}. Деякі центри стверджують, що збільшення хірургічного ризику “екстремальної резекції” може бути іноді виправданим, якщо збалансувати альтернативу трансплантації та пожиттєву імуносупресію. Але залишається багато питань щодо печінкової недостатності, обмежень для печінкової регенерації у дітей, які отримують хіміотерапію, та потенційно підвищеного ризику рецидиву пухлини, особливо у випадку з мультифокальними пухлинами \cite{pmid10466608}. 

Випадки «нерезектабельної» гепатобластоми (ГБ) внаслідок ураження всієї печінки, мультифокального враження, значного враження печінкових вен та/або ворітної вени все ще становлять 10–20\% усіх ГБ \cite{pmid14966740}. ГБ - рідкісна пухлина, яка, тим не менше, становить 75\% первинних злоякісних пухлин печінки у дітей. Рівень 5-річної виживаності дітей, уражених ГБ, які отримували комбіновану хіміотерапію на основі цисплатину та радикальну хірургічну резекцію, зараз становить 80–90\%, що як мінімум вдвічі перевищує рівень виживаності, про який повідомлялося на початку 1980-х років \cite{pmid21509775}. Незважаючи на ці захоплюючі результати, епідеміологи оцінюють 5-річне виживання без рецидиву захворювання у США не вище 50\%, припускаючи, що багато дітей поза цими дослідженнями можуть не отримувати оптимальної сучасної допомоги \cite{pmid18560935}. Керівні принципи, викладені в сучасних дослідженнях як COG, так і SIOPEL, базуються на PRETEXT. Наприклад, ті пухлини, які рекомендовані для раннього огляду  в центрі трансплантації в дослідженні COG AHEP0731, показані на рис \cite{pmid7796018}.
\subsection{Трансплантація печінки при в залежності від PRETEXT}
Система PRETEXT вперше розроблена Стратегічною групою Міжнародного наукового товариства дитячої онкології (Liver Tumor Strategy Group of the International Society of Pediatric Oncology (SIOPEL)) \cite{pmid24362406}, використовувалась SIOPEL протягом багатьох років як інструмент для оцінки ризиків. Починаючи з SIOPEL 2, пухлини групи PRETEXT I, II та III розглядали як «стандартний ризик» (СР), а PRETEXT IV, + M (метастатичні), і ті, у кого АФП <100, розглядалися  як групу «високого ризику» (ВР). Рекомендації щодо трансплантації печінки, використані в недавньому дослідженні SIOPEL 3, були такими: «Найпоширенішими причинами пухлини, яку вважають «нерезектабельною» (за винятком повної гепатектомії), є: а) пухлина чітко займає всі 4 секції печінки, за даними МРТ-сканування  ангіографії; або (b) розташування настільки близько до магістральних судин у воротах печінки та/або до печінкових вен, що малоймовірно, що буде досягнута площина резекції без пухлинного залишку (R0). Ці пацієнти повинні бути ідентифіковані під час діагностики, а їх клінічний перебіг та ведення слід ретельно спостерігати протягом першого крусу хіміотерапії спільно з хірургом-трансплантологом печінки” \cite{pmid14966739}.

У дослідженні SIOPEL-1, 12 пацієнтам зробили трансплантацію печінки, з них 7 пацієнтів отримали трансплантацію печінки як основний хірургічний метод лікування та 5 як метод порятунку. Довготривале (> 10 років) виживання без ознак пролонгації основного захворювання становило 85\% та 40\% відповідно. Всі вісім пацієнтів з пухлинами PRETEXT IV і всі шість пацієнтів з мультифокальним ГБ були вилікувані від своєї хвороби. З семи пацієнтів з макроскопічним поширенням у ворітну вену та/або печінкові вени/порожнисту вену – у 71\% спостерігали тривалу виживаність без рецидиву, а також чотири з п’яти (80\%) дітей, які мали метастази в легенях при первинному зверненні, спостерігали повне очищення легень після хіміотерапії \cite{pmid18560935}. У SIOPEL-3,  35 пацієнтам з групи високого ризику було проведено трансплантацію печінки; 33 як основний хірургічний варіант та 2 як “рятувальна” трансплантація. З 33 пацієнтів, яким була проведена первинна трансплантація, у 10 був рецидив пухлини (31\%), 8 померли від рецидиву пухлини (24\%), і рання загальна виживаність становила 75\%. З шести пацієнтів з метастатичною хворобою на момент діагностики, яким була проведена первинна трансплантація, у трьох із шести (50\%) спостерігався рецидив пухлини. Результати для груп стандартного та високого ризиків при пухлинах наведені в таблиці \cite{pmid12778356}. Результати свідчать про збільшення ГБ стандартного ризику в SIOPEL-3 в порівнянні з SIOPEL 1 і 2, і однією з причин такого поліпшення результатів є більш своєчасне і більш доцільне використання первинної трансплантації печінки при неоперабельних пухлинах протягом багатьох років \cite{pmid1323649}.

В обширному огляді зібрано 147 випадків трансплантації печінки при ГБ \cite{pmid17208562}. Двадцять вісім (19\% від загальної кількості) пацієнтів мали макроскопічне поширення у вени та 12 (8\%) - метастази в легені. Загалом 106 пацієнтам (72\%) було зроблено первинну трансплантацію, а 41 (28\%) отримали рятувальну трансплантацію - або при неповній резекції з частковою гепатектомією, або при рецидиві пухлини після попередньої часткової гепатектомії. Медіана виживаності з моменту встановлення діагнозу у таких пацієнтів становила 38 місяців (діапазон 1–121 місяць). Загальна виживаність без захворювань впродовж 6 років після трансплантації становила 82\% та 30\% для первинної трансплантації та для рятувальної трансплантації відповідно. Це становило 82\% та 71\% після трансплантації печінки від живого донору (n = 28) та після трупної трансплантації печінки (n = 119) відповідно. Багатофакторний статистичний аналіз не показав різниці щодо статі, віку та наявності метастазів у легенях на момент звернення або від виду трансплантації. Для первинних трансплантацій єдиним параметром, суттєво пов’язаним із загальною виживаністю, була макроскопічна венозна інвазія (Р = 0,045) \cite{pmid22648963}.

Досвід UNOS з трансплантації печінки при ГБ охопив 180 дітей; 140 пацієнтів (78\%) перенесли трансплантацію протягом останнього десятиліття. У середньому інтервали спостереження за пацієнтом становили 24 місяці, 1-річний, 3-річний та 5-річний показники виживання пацієнтів та алотрансплантатів становили 80\%, 72\% та 69\% та 71\%, 63\% та 61\% відповідно. Трирічні показники виживання пацієнтів при отриманні всієї печінки (63,3\%), при отримуванні долі печінки від трупного донора (21,8\%) та від живого донору (14,7\%) становили 67,4\%, 67,1\% та 84,6\% відповідно (p = 0,22). Багатофакторний аналіз визначив збіг АВО та рівню креатиніну в сироватці крові (що відображає загальний медичний стан перед трансплантацією) як єдині незалежні прогностичну показники. Що стосується типу алотрансплантата, спостерігалась тенденція до покращення виживання у підгрупі пацієнтів, які отримували алотрансплантат від живого родинного донора. Однак тенденція до покращення спостережуваних результатів може бути частково зумовлена підбором реципієнта  для трансплантації від живого донора \cite{pmid32843604}.

Reyes et al. повідомили про 12 дітей із неоперабельними ГБ, які перенесли трансплантацію в Пітсбурзі у період з травня 1989 р. до грудня 1998 р. Показники виживання після трансплантації  1-річні, 3-річні та 5-річні становили 92\%, 92\% та 83\% відповідно. Внутрішньовенна інвазія, наявність mts в грудні лімфатичні вузли та суміжне поширення не мали значного негативного впливу на результати; віддалені метастази спричинили дві смерті \cite{pmid20345611}.

Серед 52 дітей (<15 років), які отримували лікування (1978–2003 рр.) у клініці Cliniques Saint-Luc, Брюссель, часткова гепатектомії була виконана 39 пацієнтам, а трансплантацію - 13 пацієнтам. Усі пацієнти, включаючи трансплантованих, отримували хіміотерапію відповідно до протоколів SIOPEL. Загалом виживання без рецидиву становило 80\% та 89\% відповідно. Частота рецидивів становила 23\% та 7,6\% відповідно \cite{pmid20406943}.
У Бірмінгемі, Великобританія, із 34 дітей з ГБ, які отримували лікування протягом 10 років (1991–2000 рр.), 12 пацієнтам зробили первинну трансплантацію печінки, оскільки пухлина залишалася нерезектабельною після хіміотерапії, а 2 пацієнти отримали рятувальну (екстрену) трансплантацію у зв’язку з  рецидивом після часткової гепатектомії. Рівень виживання без рецидиву становив 100\% після первинної трансплантації та 50\% у пацієнтів із рятувальною трансплантацією. Автори дійшли висновку, що трансплантація є потенційно лікувальним варіантом при нерезектабельних ГБ, за умов хіміочутливості (зменшення альфа-фетопротеїну та зменшення розміру пухлини) в той час як пацієнти з рецидивуючим або стійким захворюванням не є гарними кандидатами для трансплантації \cite{pmid20070564}.

У лікарні King’s College Hospital, Лондон, Великобританія, ортотопічна трансплантація печінки була виконано 25 дітям, яким було встановлено діагноз нерезектабельна ГБ. П'ятнадцять пацієнтів знаходились на рівні IV за визначенням важкості захворювання до лікування (PRETEXT IV) та 10 були оцінені як III (PRETEXT III). Передопераційна хіміотерапія проводилась відповідно до системи стратифікації ризиків для дітей із протоколами ГБ SIOPEL. Вісімнадцять отримали трупні трансплантати, а сім перенесли трансплантацію печінки від живого родинного донору. Виживання пацієнтів та трансплантатів після трупної трансплантації становило 91\%, 77,6\% та 77,6\% відповідно на 1, 5 та 10 років. Виживання пацієнтів та трансплантатів у дітей, які перенесли трансплантацію від живого родинного донора, становило 100\%, 83,3\% та 83,3\% відповідно на 1, 5 та 10 років. Усі діти, які вижили, крім одного, залишаються вільними від рецидивів із середнім спостереженням 6,8 років (діапазон: 0,9–14,9).
Зафіксовано п’ять смертей в середньому через 13 місяців після трансплантації, у чотирьох пацієнтів вони пов’язані із рецидивом захворювання та через дихальну недостатність у одного пацієнта \cite{pmid16794509}.

Група Далласа повідомила про дев'ять реципієнтів з діагнозом нерезектабельна ГБ, які отримали трансплантацію. Був один епізод рецидиву. Виживання без рецидиву становило 66\% (медіана спостереження: 7,7 років). Усі реципієнти отримували передопераційну хіміотерапію та 67\% отримували післяопераційну хіміотерапію. Єдиний випадок, коли рівні АФП не знижувались до низьких або невизначуваних рівнів після трансплантації, був у пацієнта з рецидивуючою пухлиною \cite{pmid10839879}.

Група "Омаха" повідомила про десять випадків нерезектабельних ГБ, які отримували як лікування методом трансплантації печінки (1985–2003) (вісім трансплантацій від трупних донорів, дві від живих родинних донорів). Хіміотерапію до трансплантації застосовували у 90\% випадків. Виживання після трансплантації коливається від 3,7 років до 18,6 років. Троє пацієнтів померли від рецидиву захворювання на 4, 14 та 38 місяцях. Два реципієнти після трансплантації від живого родинного донору змогли отримати передтрансплантаційну хіміотерапію з швидким рішенням щодо трансплантації; обидва вони живі і здорові через 5,5 і 11 років після трансплантації \cite{pmid2544067}. 

У Кіото трансплантацію печінки від живого донору було проведено 14 пацієнтам з нерезектабельними ГБ (PRETEXT III: 7; PRETEXT IV: 7), як терапію відчаю. Рівні виживання 1 та 5 років та виживання трансплантата становили 78,6\% та 65,5\%. Четверо дітей померли від рецидиву пухлини. Поганими прогностичними факторами були макроскопічна венозна інвазія та екстра-ураження печінки, які не пережили 5 років \cite{pmid17430157}.

\subsection{Керівні принципи при трансплантації при ГБ}
Ряд керівних принципів слід враховувати при розгляді питання про трансплантацію при ГБ. 

\subsubsection{Відповідь на хіміотерапію} 
Лікування усіх пацієнтів потрібно починати з неоад'ювантної хіміотерапієї \cite{pmid12352881}. Передопераційна хіміотерапія робить більшість пухлин меншими, краще відокремленими від навколишньої тканини печінки та, швидше за все, повністю резектабельними. Більшість реакцій на хіміотерапію, відбувається протягом перших двох циклів хіміотерапії з прогресивним плато у відповідь на них \cite{pmid12778356}. Слід уникати тривалих курсів передопераційної хіміотерапії, коли пухлина є нерезектабельною, через зменшуючий вплив на пухлину в поєднанні зі значним ризиком викликати стійкість до хіміотерапії при тривалому впливі \cite{pmid25649007}. Хоча нам все ще бракує доказів, слід, ймовірно, рекомендувати хіміотерапію після трансплантації, якщо існує обґрунтована ймовірність того, що будь-який мікроскопічний залишок залишається хіміочутливим.

\subsubsection{АФП <100} 
АФП <100– це поганий прогностичний фактор. АФП є надійним маркером у більшості випадків із підвищеним значенням, а реакція на хіміотерапію з тенденцією до зменшення АФП має хорошу прогностичну цінність. Навпаки, низький рівень АФП (<100) при первинному зверненні є предиктором високого ризику \cite{pmid1328586}; швидше за все, в такому випадку найкраще уникати трансплантації.

\subsubsection{PRETEXT для визначення потенційної потреби в пересадці} 
Показано, що PRETEXT є корисним способом допомогти ідентифікувати пацієнтів як під час першого звернення, так і після хіміотерапії, яким знадобиться повна гепатектомія та трансплантація. Ідентифікація під час первинного звернення дозволяє своєчасно звернутися до лікаря та провести повну оцінку у спеціалізованому центрі печінки з можливістю трансплантації \cite{pmid1323649}. Раннє направлення потенційно неоперабельної ГБ до хірургічної бригади з можливістю трансплантації печінки рекомендують усі три основні стратегічні групи пухлини печінки: COG, SIOPEL та GPOH \cite{pmid18970927}. Спрощений варіант рекомендацій щодо направлення на трансплантацію, що використовується у поточному протоколі COG, AHEP0731, показаний на рис.

\subsubsection{Мультифокальні пухлини} 
Мультифокальність може бути поганим прогностичним фактором \cite{pmid28126357}. Більшість великих мультифокальних пухлин можуть мати приховані мікроскопічні сателітні ураження і потенційно не піддаються резекції. 

\subsubsection{Метастатична хвороба} 
ГБ поширюється завдяки судинному шляху метестазування, як правило, в легені. Більшість центрів вважають, що пацієнтів з метастазами в легенях не слід виключати з листа очікування на трансплантацію печінки, якщо метастази зникають або зменшуються після хіміотерапії, варто завершити їх лікування хірургічною резекцією залишків, якщо це необхідно \cite{pmid22201955}.

\subsubsection{Макроваскулярна інвазія} 
Макроваскулярна венозна інвазія є поширеним явищем при великих розмірах ГБ. Хоча це може негативно вплинути на прогноз \cite{pmid12352881}, її наявність не повинна суперечити трансплантації \cite{pmid20406943}. Пухлини з персистуючою великою венозною інвазією після перших двох циклів хіміотерапії потенційно не піддаються обробці. Чинний протокол Дитячої онкологічної групи (COG) рекомендує на ранньому етапі співпрацювати зі спеціальною групою печінкових хірургів, здатною забезпечити комплексну резекцію та негайну наявність трансплантації, щоб усі плани остаточної операції були готові до кінця четвертого циклу неоад'ювантної хіміотерапії.

\subsubsection{Трансплантація печінки від живого донора проти трупного алотрансплантата} 
Існує тенденція до вищого виживання пацієнтів без рецидиву у дітей, які отримують трансплантацію печінки від живого родинного донора \cite{pmid31155201}. Коли доступний живий донор, можна призначити оптимальну хіміотерапію до трансплантації, з швидким рішенням щодо трансплантації \cite{pmid10839879}.

\subsection{Рекомендації щодо трансплантації печінки при ГБ}
Наступні рекомендації були розроблені протягом багатьох років і в даний час рекомендовані COG, SIOPEL та GPOH. Важливо, щоб консультація з центром трансплантації, що має спеціальний досвід у галузі педіатричної хірургії печінки, розглядалася на початку лікування, щоб запобігти затримкам та небажаним тривалим курсам хіміотерапії в очікуванні резекції та трансплантації \cite{Chen2019}.

\subsubsubsection{Мультифокальні вогнища PRETEXT IV} 

Мультифокальний PRETEXT IV при ГБ за відсутності будь-якого метастатичного захворювання після хіміотерапії (POST-TEXT – M) є чітким показом до трансплантації печінки. Видиме видалення пухлини з одного відділу печінку (мультифокальний PRETEXT IV - мультифокальний POST-TEXT III) не повинно відволікати через високу ймовірність постійних мікроскопічних життєздатних неопластичеих клітин у теперішньому рентгенологічно “чистому” зрізі. Клініцисти повинні протистояти спокусі посилити (кількість курсів) хіміотерапію з ціллю уникнення трансплантації. Цих пацієнтів слід лікувати за протоколами хіміотерапії з високим ризиком, як і пацієнти з локалізованими пухлинами, що піддаються частковій гепатектомії, з такою ж кількістю циклів хіміотерапії до та після трансплантації, так і пацієнти, які перенесли часткову гепатектомію. Багато з цих пацієнтів з великими мультифокальними пухлинами перенесли хіміотерапію до POST-TEXT III і перенесли трисегментектомію, і лише при рецидиві звернувшись для  трансплантації. Попередній досвід задокументував погіршення результатів, досягнутих у більшості пацієнтів із „рецидивом”, замість первинної трансплантації. Аналогічним чином було показано, що резекція одиночної первинної пухлини з неанатомічною клиноподібною резекцією сателітних вузлів несе високий ризик місцевого рецидиву \cite{pmid19040959}.

\subsubsection{Одиночне вогнище PRETEXT – IV} 

Первинна трансплантація печінки - це найкращий варіант для великого, одиничного, PRETEXT IV при ГБ, що включає всі чотири відділи печінки, якщо тільки після неоад’ювантної хіміотерапії не буде продемонстровано зниження пухлини до однофокального POST-TEXT III. Якщо у такому випадку чітка ретракція пухлини від анатомічної межі одного латерального сектора дозволила б виконати трисегментектомію \cite{pmid31683629}.

\subsubsection{PRETEXT III + V, + P} 

У підгрупі PRETEXT III при ГБ -  велика судинна інвазія може унеможливити стандартну трисегментектомію. Резекція в умовах серйозної венозної інвазії ризикує залишити життєздатну неопластичну (пухлинну) тканину, якщо хірург повинен відшарувати життєздатну пухлину безпосередньо з ураженої вени. Деякі аргументовано виступають за резекцію та реконструкцію вен на відміну від трансплантації \cite{pmid20938901}. Немає досліджень, які б порівнювали результати часткової резекції з великою венозною дисекцією та повною резекцією з трансплантацією. Часткова резекція підвищує ризик хірургічних ускладнень, включаючи кровотечу та/або перешкоду притоку або відтоку від вен, а також позитивний край резекції пухлини (R1). Важливість чистоти зрізу залишається дискусійною, оскільки мікроскопічна позитивна межа зрізу після хіміотерапії та хірургічного втручання не завжди може вплинути на прогноз \cite{pmid24757164}.

\subsubsection{Операції в спеціалізованих центрах} 

Усі пацієнти з такими типами враження на початку лікування потрібно направляти до хірургічної бригади, яка має досвід як радикальної резекції, так і трансплантації печінки. В руках такої команди, якщо велика венозна резекція призводить до компрометації печінки, команда повинна бути готова перейти безпосередньо до трансплантації. Рішення про те, яка форма терапії може бути найкращою за певних обставин, може приймати хірургічна команда, яка має досвід і здатність робити і те, і інше.
Макроскопічна венозна інвазія (ворітна вена, печінкові вени, порожниста вена) є лише відносним протипоказанням, якщо можна здійснити повну резекцію інвазованих венозних структур. Якщо є докази або підозри на проростання в запечінковий відділ порожнистої вени, її слід видалити “одним блоком” та реконструювати \cite{pmid20345611}. Огляд світового досвіду показує, що венозна інвазія асоціюється із значно коротшим виживанням (Р = 0,045) \cite{pmid10466608}. На відміну від них, 71\% цих пацієнтів були живими та не зазнали рецидиву захворювання більше 10 років після трансплантації печінки у дослідженні SIOPEL l. З дев'яти пацієнтів з TNM IVA/IVB (вісім з великою внутрішньогепатичною венозною інвазією), про які повідомляли Рейес та його співробітники, семеро були живими та без рецидиву через 21–146 місяців після трансплантації \cite{pmid21509775}.

\subsubsection{Легеневі метастази} 
Абсолютним протипоказанням до трансплантації печінки є стійкі легеневі метастази, які не реагують на неоад’ювантну хіміотерапія і не піддаються хірургічній резекції. Пухлина повинна виявляти принаймні часткову реакцію на хіміотерапію (зменшення розміру пухлини, зменшення вмісту АФП у сироватці крові та зменшення розміру або зникнення легеневих вузлів). Стабільна або прогресуюча хвороба є відносним протипоказанням до трансплантації \cite{pmid16045186}. Метастази в легенях, які не повністю зменшуються в розмірах/зникають після хіміотерапії, повинні бути хірургічно видалені та підтверджені гістологічним діагнозом. Деякі виступали за стернотомію та двосторонню пальпацію легенів, а не за односторонню клинову резекцію сателітних вузлів до трансплантації, хоча це залишається суперечливим \cite{pmid11819207}. Легеневі метастази, які повністю зникають при хіміотерапії з хірургічною резекцією або без неї, не є протипоказанням, проте ризик рецидиву у легенях після трансплантації є значним, і тому використання трансплантації печінки для дітей з метастатичними захворюваннями залишається суперечливим \cite{pmid24852330}.

\subsubsection{«Рятувальна» трансплантація при рецидиві}

Кілька серій показали кращий результат після первинної трансплантації (загальна виживаність близько 80\%) порівняно з “рятувальною” трансплантацією (приблизно 30–40\% загальної виживаності) \cite{pmid15285242}. Підставою для цього, безсумнівно, є багатофакторність, але двома важливими причинами є ймовірність стійкості до хіміотерапії при рецидивних пухлинах та ослаблений стан пацієнтів при пересадці перед термінальною стадією захворювання. Потенційні кандидати для трансплантації вимагають не тільки ретельної оцінки свого основного захворювання, але й ретельного врахування їх здатності переносити фізіологічний стрес трансплантації. Доксорубіцин є кардіотоксичним, а цисплатин - нефротоксичним. Детальна ехокардіограма та оцінка функції нирок мають важливе значення перед трансплантацією, особливо «рятувальною» трансплантацією. Через місяці, іноді роки невдалої терапії, соматичний стан дитини може бути порушений, що робить її більш сприйнятливими до інфекційних ускладнень \cite{pmid15285242}.

\subsection{Післятрансплантаційна імуносупресія}

Існує занепокоєння щодо довгострокових наслідків поєднаної нефротоксичності цисплатину та інгібіторів кальциневрину. Діти, які отримують хіміотерапію, певною мірою соматично пригнічені на момент трансплантації; їм може знадобитись спеціальне зменшення інгібіторів кальциневрину після трансплантації. У серії, опублікованій брюссельською групою \cite{pmid18444949}, 12 дітей з первинною трансплантацією порівнювали з такою ж когортою з 12 дітей, яким пересадили печінку при не злоякісній первинній патології за той же період часу. Відповідні критерії включали вік, стать, тип трансплантата (десять трансплантатів від живого донору та два трупних трансплантата в обох групах) та імуносупресивні режими (такролімус-стероїди: п’ять і п’яти; такролімус-базиліксимаб: чотири і чотири; такролімус монотерапія: три і три). Загальна виживаність пацієнтів становила 91\% у першій групі (1 пацієнт помер від рецидиву пухлини; інші 11 - живі та не зазнали рецидиву) та 100\% у другій групі. Рівень виживання без відторгнення до 5 років після трансплантації становив 91\% та 58\% відповідно (р: 0,079), незважаючи на значно нижчий мінімальний рівень такролімусу в крові через 90 днів (р: 0,004), 6 місяців (р: 0,034) та 1 рік (р: 0,019) у хворих на ГБ \cite{pmid16185597}. 

Цей обмежений досвід свідчить про те, що зниження мінімальних рівнів такролімусу в крові, що застосовуються у пацієнтів, яким виконали трансплантацію при ГБ (і, ймовірно, інших пухлин, які лікуються хіміотерапією), з метою захисту нирок від кумулятивної нефротоксичності інгібіторів кальциневрину та цисплатину, не збільшує частоту відторгнення протягом першого року після трансплантації \cite{pmid20223320}.


\subsection{Висновки}

Тотальна гепатектомія та трансплантація печінки повинні розглядатися як невід’ємна частина сучасного лікування у дітей із високим ризиком ГБ. Використання хіміотерапії для зменшення розміру або ступеня поширеності пухлин у цих дітей з високим ризиком піддає їх ризику підвищеної захворюваності, більш високому рівню рецидивів пухлини або смерті до або під час резекції. 

Незважаючи на те, що альтернативна терапія за допомогою “рятувальної” хірургії повідомляла про хороші результати в деяких випадках, вона все ще залежить від спеціалізованих хірургічних навичок та хірургічних бригад з великим досвідом. Саме ці спеціалізовані хірургічні бригади найкраще підходять для прийняття рішення щодо трансплантації. 

Пацієнти з метастатичними захворюваннями все ще можуть отримати користь від лікування, включаючи трансплантацію, але залишаються значні питання щодо їх оптимального лікування. 
