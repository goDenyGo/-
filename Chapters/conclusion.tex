\chapter{АНАЛІЗ ТА УЗАГАЛЬНЕННЯ РЕЗУЛЬТАТІВ ДОСЛІДЖЕННЯ}

\textbf{Тут повинні бути висновки, які є в кінці кожного розділу:}

ТЕКСТ ПРИБЛИЗНИЙ - тут тільки основна ідея!!!

Проведений аналіз літературних джерел дозволив встановити, що є велика кількість нерезектабельних пацієнтів яким можливо виконати трансплантацію

Загальна частота виникнення гепатобластоми скаладає .... 

Ефективність неоадьювантної поліхіміотерапії ....

Резектабельнісь визначається ....

Для виконання поставлених задач нами було досліджено .....

Тактика лікування гепатобластоми включає ....

Особливістю резекційного методу є необхідність судинних реконструкцій у випадках судинних інвазій .....

Особливостями трансплантаційного методу є .....

Запропонований алгоритм із залученням запропонованих способів дозволяє отримати ......


\chapter{ВИСНОВКИ}
\begin{enumerate}
    \item При первинній діагностиці в 32,1\% випадків спостерігали неоперабельні форми гепатобластоми. Проведення поліхіміотерапії  дозволило отримати конверсію в операбельні форми у 95\% з цих пацієнтів, та підвищити загльний рівень операбельності до 97\%.
    
    \item З усіх відомих гістологічних форм частіше (у 55\% випадків) зустрічали епітеліальний та (у 41,1\%) змішаний тип гепатобластоми. Наявність у пацієнта дрiбноклiтинного недиференцiйованого та макротрабекулярного пiдтипів та пов'язана із погіршенням прогнозу та зменшенням післяопераційної виживаності до \%.
    
    \item Вибір резекційного або трансплантаційного методу хірургічного лікування визначається резектабельністю пухлини, яка залежить від ступеня її поширеності за PRETEXT та наявності інвазії в магістральні судини. У 67,3\% пацієнтів з резектабельними формами гепатобластоми були виконані обширні резекції печінки, а у 10,0\% з нерезектабельними - трансплантація частини печінки від живого родинного донора із використанням розроблених способів венозної реконструкції.
    
    \item Резекційний метод хірургічного лікування гепатобластоми включав виконання розширених (у \% пацієнтів) та обширних (у \% пацієнтів) резекцій печінки із пластикою нижньої порожнистої вени в \% та портопластикою в \%.
    
    \item Транслантаційний метод хірургічного лікування гепатобластоми включав трансплантацію лівої латеральної секції печінки у 7 (у 77,8\% пацієнтів), лівої долі печінки у 1 (у 11,1\% пацієнта) від живого родинного донора та у 1 (у 11,1\% пацієнта) пацієнта тpaнcплaнтaцiя лiвoï лaтepaльнoï ceкцiï пeчiнки вiд живoro poдиннoгo дoнopa та мyльтивicцepaльнa peзeкцiя .
    
    \item Впровадження запропонованої тактики вибору резекційного або трансплантаційного методу хірургічного лікування дозволило збільшити кількість операбельних форм гепатобластоми на  100500\% за рахунок виконання трансплантації печінки нерезектабельним пацієнтам, та отримати задовільні показники післяопераційної морбідності, летальності та віддаленої виживансоті на рівні   \% \% та \% відповідно.
    
\end{enumerate}


\chapter{Висновки марата}
\begin{enumerate}
\item Резекційні та трансплантаційні технології є ефективними методами хірургічного лікування ге- патобластоми у дітей та у поєднанні з хіміотерапією дозволяють отримати хороший довгостроковий ре- зультат.
\item  Найближчі та віддалені результати після транс- плантації печінки кращі, ніж після резекції печінки, навіть попри те, що у пацієнтів трансплантаційної групи спостерігається більший об’єм пухлинного ураження печінки.
\item  Трансплантація печінки є єдиним ефективним ме- тодом хірургічного лікування у дітей з нерезектабельною гепатобластомою, що показує хороші результати загаль- ної та безрецидивної виживаності.
\end{enumerate}