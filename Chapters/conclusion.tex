\chapter{АНАЛІЗ ТА УЗАГАЛЬНЕННЯ РЕЗУЛЬТАТІВ ДОСЛІДЖЕННЯ}

\textbf{Тут повинні бути висновки, які є в кінці кожного розділу:}

ТЕКСТ ПРИБЛИЗНИЙ - тут тільки основна ідея!!!

Проведений аналіз літературних джерел дозволив встановити, що є велика кількість нерезектабельних пацієнтів яким можливо виконати трансплантацію

Загальна частота виникнення гепатобластоми скаладає .... 

Ефективність неоадьювантної поліхіміотерапії ....

Резектабельнісь визначається ....

Для виконання поставлених задач нами було досліджено .....

Тактика лікування гепатобластоми включає ....

Особливістю резекційного методу є необхідність судинних реконструкцій у випадках судинних інвазій .....

Особливостями трансплантаційного методу є .....

Запропонований алгоритм із залученням запропонованих способів дозволяє отримати ......


\chapter{ВИСНОВКИ}
\begin{enumerate}
    \item При первинній діагностиці в \% випадків спостерігали неоперабельні форми гепатобластоми. Проведення поліхіміотерапії за схемою PLADO дозволило отримати конверсію в операбельні форми у \% з цих пацієнтів, та підвищити загльний рівень операбельності до  \%.
    
    \item З усіх відомих гістологічних форм частіше (у \% випадків) зустрічали ...... та .... форми гепатобластоми. Наявність у пацієнта ..... форми була визначена в \% випадків та пов'язана із погіршенням прогнозу та зменшенням післяопераційної виживаності до \%.
    
    \item Вибір резекційного або трансплантаційного методу хірургічного лікування визначається резектабельністю пухлини, яка залежить від ступеня її поширеності за PRETEXT та наявності інвазії в магістральні судини. У \% пацієнтів з резектабельними формами гепатобластоми були виконані обширні резекції печінки, а у \% з нерезектабельними - трансплантація частини печінки від живого родинного донора із використанням розроблених способів венозної реконструкції.
    
    \item Резекційний метод хірургічного лікування гепатобластоми включав виконання розширених (у \% пацієнтів) та обширних (у \% пацієнтів) резекцій печінки із пластикою нижньої порожнистої вени в \% та портопластикою в \%.
    
    \item Транслантаційний метод хірургічного лікування гепатобластоми включав трансплантацію лівої латеральної секції печінки (у \% пацієнтів) та лівої долі печінки (у \% пацієнтів) від живого родинного донора із використанням запропонованих способів венозної реконструкції в \%.
    
    \item Впровадження запропонованої тактики вибору резекційного або трансплантаційного методу хірургічного лікування дозволило збільшити кількість операбельних форм гепатобластоми на  \% за рахунок виконання трансплантації печінки нерезектабельним пацієнтам, та отримати задовільні показники післяопераційної морбідності, летальності та віддаленої виживансоті на рівні   \% \% та \% відповідно.
    
\end{enumerate}