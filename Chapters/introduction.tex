\chapter{ВСТУП}
\textbf{Актуальність теми}. Поширеність гепатобластоми становить до 1,5 випадків на 1 млн. дітей і визначається тенденція щорічного збільшення захворюваності гепатобластоми на 4,3\% Гепатобластома становить 60-85\% усіх злоякісних пухлин печінки у дітей. У 95\% випадків діагноз «гепатобластома» встановлюють у віці до 4-х років. Білобарне ураження печінки визначається в 20\% - 30\% випадків і мультицентричне ураження печінки в 15\% випадків. У 25-35\% випадків гепатобластома резистентна до хіміотерапії. При відсутності лікування - 100\% смертність протягом 8-16 місяців.

\textbf{Метою дослідження} є: покращити результати хірургічного лікування дітей з гепатобластомою

Відповідно до поставленої мети сформульовані наступні \textbf{завдання дослідження}:
\begin{enumerate}
    \item Вивчити локалізацію, ступінь поширеності гепатобластоми у дітей, а також оцінити результати неоад'ювантної поліхіміотерапії.
    \item Вивчити патоморфологічні особливості гепатобластоми.
    \item Розробити оптимальний алгоритм діагностики, визначити критерії резектабельності і терміни виконання резекційних та трансплантаційних способів лікування гепатобластоми в залежності від стадії пухлини по PRETEXT і групи ризику по CHICS.
    \item Розробити способи резекційного та трансплантаційного методів лікування гепатобластоми в залежності від стадії пухлини.
    \item Вивчити найближчі та віддалені результати резекційного та трансплантаційного способів лікування гепатобластома в залежності від стадії PRETEXT, гістологічного типу гепатобластоми, групи ризику по CHICS і оцінки відповіді пухлини на хіміотерапію по RECIST.
    
\end{enumerate}


\textbf{Об’єкт дослідження:} Гепатобластома

\textbf{Предмет дослідження:} Хірургічне лікування пацієнтів з гепатобластомою

\textbf{Методи дослідження:} клініко–лабораторні, інструментальні методи (ультразвукове дослідження, мультидетекторна спіральна комп’ютерна томографія, магніторезонансна томографія, волюметрія печінки та черевної порожнини, зовнішнє вимірювання розмірів черевної порожнини), статистичні.


\textbf{Наукова новизна отриманих результатів.} 
\begin{enumerate}
    \item В процесі подальшої роботи будуть розроблені нові методи резекційного та трансплантаційного способів лікування гепатобластоми;

    \item За період дослідження буде дана порівняльна характеристика та віддалені результати трансплантаційних і резекційних способів лікування хворих з гепатобластомою.
\end{enumerate}

\textbf{Практичне значення отриманих результатів.} 
Запропонований в процесі роботи алгоритм вибору оперативного втручання дозволив поліпшити результати лікування дітей з гепатобластомою та  розширити покази до оперативного втручання за рахунок виконання трансплантації печінки у нерезектабельних пацієнтів. Обраний підхід до лікуванння пацієнтів з обширними формами ураження гепатобластоми дозволив знизити морбідність та ранню післяопераційну летальність, поліпшити якість життя пацієнтів.

\textbf{Апробація роботи}. 
Апробація роботи. Основні матеріали і положення НДР докладені и обговорені на V з'їзді трансплантологів (5-7 жовтня 2011 р., м.Харків) 

\textbf{Публикации} 
По темі НДР опубліковано 21  наукова работа (8 в журналах, рекомендованих ВАК України, 4 в зарубіжних журналах), отримано 4 авторських свідоцтва.
Об'єм роботи. 

\textbf{Об'єм роботи}. 
Звіт виконаний на \pageref{LastPage} сторінці печатного тексту. Складається з вступу, 5 розділів, заключення, висновків і практичних рекомендацій, ілюстрована \tottab\ таблицями, \totfig\ малюнками. Список літератури складається з \total{citenum} джерел. Представлена работа є результатом комплексного дослідження данних отриманих в ході виконання резекцій та трансплантацій печінки пацієнтам с гепатобластомою, що знаходились на лікуванні у відділенні хірургії та трансплантації печінки НІХТ ім. акад. О.О.Шалімова.  за період  с 2001 по 2021 роки.
Обстеження проводилось в динаміці на наступних етапах: в передопераційному періоді, під час операції, в найближчому і віддаленому післяопераційному періоді.

\textbf{Впровадження в практику}. 
Впровадження в практику. Результати роботи впроваджені в відділенні хірургії і трансплантації печінки Национального інституту хірургії і трансплантології ім. акад. О.О.Шалімова.
Робота виконана в відділенні хірургії і трансплантації печінки Национального інституту хірургії і трансплантології ім. акад. О.О.Шалімова.