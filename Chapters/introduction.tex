\chapter{ВСТУП}
\textbf{Актуальність теми}. Поширеність гепатобластоми становить до 1,5 випадків на 1 млн. дітей і визначається тенденція щорічного збільшення захворюваності гепатобластоми на 4,3\% Гепатобластома становить 60-85\% усіх злоякісних пухлин печінки у дітей. У 95\% випадків діагноз «гепатобластома» встановлюють у віці до 4-х років. Білобарне ураження печінки визначається в 20\% - 30\% випадків і мультіцентричне ураження печінки в 15\% випадків. У 25-35\% випадків гепатобластома резистентна до хіміотерапії. При відсутності лікування - 100\% смертність протягом 8-16 місяців.

\textbf{Метою дослідження} є: покращити результати хірургічного лікування дітей з гепатобластомою

Відповідно до поставленої мети сформульовані наступні \textbf{завдання дослідження}:
\begin{enumerate}
    \item Вивчити локалізацію, ступінь поширеності гепатобластоми у дітей, а також оцінити результати неоад'ювантної поліхіміотерапії.
    \item Вивчити патоморфологічні особливості гепатобластоми.
    \item Розробити оптимальний алгоритм діагностики, визначити критерії резектабельності і терміни виконання резекційних та трансплантаційних способів лікування гепатобластоми в залежності від стадії пухлини по PRETEXT і групи ризику по CHICS.
    \item Розробити способи резекційного та трансплантаційного методів лікування гепатобластоми в залежності від стадії пухлини.
    \item Вивчити найближчі та віддалені результати резекційного та трансплантаційного способів лікування гепатобластома в залежності від стадії PRETEXT, гістологічного типу гепатобластоми, групи ризику по CHICS і оцінки відповіді пухлини на хіміотерапію по RECIST.
    
\end{enumerate}


\textbf{Об’єкт дослідження:} Гепатобластома

\textbf{Предмет дослідження:} Хірургічне лікування пацієнтів з гепатобластомою

\textbf{Методи дослідження:} клініко–лабораторні, інструментальні методи (ультразвукове дослідження, мультидетекторна спіральна комп’ютерна томографія, магніторезонансна томографія, волюметрія печінки та черевної порожнини, зовнішнє вимірювання розмірів черевної порожнини), статистичні.


\textbf{Наукова новизна отриманих результатів.} 
\begin{enumerate}
    \item В процесі подальшої роботи будуть розроблені нові методи резекційного та трансплантаційного способів лікування гепатобластоми ;

    \item За період дослідження буде дана порівняльна характеристика та віддалені результати трансплантаційних і резекційних способів лікування хворих з гепатобластомою.
\end{enumerate}

\textbf{Практичне значення отриманих результатів.} 
Запропоновані в процесі роботи оперативні втручання дозволять поліпшити результати лікування дітей з гепатобластомою:
\begin{enumerate}
    \item знизити летальність;
    
    \item знизити морбідность;
    
    \item знизити кількість післяопераційних ускладнень;
    
    \item поліпшити якість життя.
\end{enumerate}

\textbf{Апробація роботи}. 
Апробация работы. Основные материалы и положения НИР доложены и обсуждены на V съезде трансплантологов (5-7 октября 2011 г., г.Харьков) 

\textbf{Публикации} 
По теме НИР опубликовано 21  научная работа (8 в журналах, рекомендованых ВАК Украины, 4 в зарубежных журналах), получено 4 авторских свидетельства.
Объем работы. 

\textbf{Об'єм роботи}. 
Диссертация выполнена на 131 странице печатного текста. Состоит из введения, 6 разделов, заключения, выводов и практических рекомендаций, иллюстрирована 28 таблицами, 33 рисунками. Список литературы содержит 6 отечественных и 171 зарубежный источник. Представленная работа является результатом комплексного исследования рентген анатомии 39 донора левой латеральной секции печени,  функционального  состояния трансплантата и гепатоспланхнической гемодинамики у 38 педиатрических реципиентов левой латеральной секции печени, находившихся на лечении в отделении хирургии и трансплантации печени НИХиТ им. акад. А.А.Шалимова.  за период  с 2004 по 2011 годы.
Обследование проводилось в динамике на следующих этапах: в исходном дооперационном периоде, во время операции, в ближайшем и отдаленном послеоперационном периодах.

\textbf{Впровадження в практику}. 
Внедрение в практику. Результаты работы внедрены в отделении хирургии и трансплантации печени Национального института хирургии и трансплантологии им. акад. А.А.Шалимова.
Работа выполнена в отделе хирургии и трансплантации печени Национального института хирургии и трансплантологии им. акад. А.А.Шалимова.
